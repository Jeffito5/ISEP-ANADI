\subsection{Exercício 2}
Face à dimensão e quantidade de dados que temos, é comum encontrarmos dados com valores que não nos permitam estudá-los, sendo necessária a sua remoção do dataset. O pressuposto do exercício 2 é a execução do pré-processamento dos dados. Para dar início a este exercício, recorremos à biblioteca \textbf{na}, mais concretamente à função \textbf{omit}, para agilizar a remoção de dados não validos - \textbf{NA} -  para estudo do nosso data frame \textit{Clientes\_DataSet.csv}. Após análise do estado corrente do data frame, identificamos as colunas com valores numéricos que não fazem sentido semanticamente serem iguais ou inferiores a zero. As colunas identificadas foram: Fidelização, TarifaMensal, TotalTarifas. Relativamente aos outliers, fizemos a sua identificação através da função \textbf{boxplot}, recorrendo mais precisamente à propriedade \textbf{out} para obter o números de outliers. Nas três colunas não foram identificados outliers. Para concluir o exercício, efetuamos a remoção do \textit{ClienteID}, uma vez que a identificação do cliente não é relevante para o estudo em questão.