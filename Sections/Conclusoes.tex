\section{Conclusões}
Recorrendo aos dados fornecidos no contexto de uma operadora telefónica, mais concretamente o registo de serviços ativos, fidelizações, tarifas etc, foi-nos possível desenvolver uma análise dos dados, tendo como base os conceitos teóricos lecionados na unidade curricular. Após a análise, concluímos que:

\begin{itemize}

\item Recorrendo aos dados fornecidos no contexto de uma operadora telefónica, mais concretamente o registo de serviços ativos, fidelizações, tarifas etc, foi-nos possível desenvolver uma análise dos dados, tendo como base os conceitos teóricos lecionados na unidade curricular. A utilização destes dados permitiu-nos então o estudo de algoritmos mais avançados, nomeadamente algoritmos de regressão linear simples, regressão linear múltipla, árvores de decisão, k vizinhos mais próximos e redes neuronais, sendo este algoritmos divididos em duas grandes áreas, Regressão e Classificação.

\item Na primeira parte (Regressão) do trabalho foi-nos possível perceber de maneira concreta a correlação que existe entre variáveis, nomeadamente no exercício 3 e 4, onde neste ultimo conseguimos observar a reta de regressão da tarifa fase à fidelização. 

\item Do exercício 5 a 7 aplicamos algoritmos de regressão linear múltipla, árvore de regressão e de rede neuronal, onde utilizamos a técnica de hold out para criar os nossos modelos com dados de treino e validamos o erro nos nossos modelos com dados de testes, percebendo assim o quão eficaz os modelos são. Concluímos que as técnicas da árvore de regressão e rede neuronal são as mais eficazes. Por fim, recorrendo à média, conseguimos então distinguir os dois melhores modelos, validando no final com testes de hipóteses (t.test e wilcox.test). Reparamos que existe uma diferença significativa entre a árvore de regressão e a rede neuronal, sendo esta útlima a técnica mais eficaz como verificamos através dos valores MAE e RMSE. Também concluímos que, uma rede neuronal com mais nós, apesar do tempo de criação, tem valores de MAE e RMSE menores. Contudo, o grupo optou por utilizar apenas 1 nó.

\item Para finalizar, na última parte do trabalho (Classificação), o estudo foi efetuado ainda com a técnica de hold out, mas com o foco desta vez em árvore de decisão, rede neuronal e k vizinhos mais próximos, onde procurávamos avaliar as medidas de desempenho dos diferentes modelos. Concluímos que as técnicas da árvore de decisão e K-vizinhos-mais-próximos tiveram uma maior percentagem de accuracy. Novamente recorrendo à média, distinguimos os dois melhores modelos com testes de hipóteses (t.test e wilcox.test). Contudo, ao contrário do exercício 7, estes 2 modelos não tinham uma diferença significativa. Para uma melhor observação, fizemos um boxplot e concluímos que a técnica da árvore de decisão teve um desempenho ligeiramente superior aos K-vizinhos-mais-próximos. 

\item Por ultimo, no exercício 11, fizemos uma avaliação das medidas de desempenho, a qual é bastante demorada dado a dimensão dos dados e a complexidade algorítmica, sobre a Accuracy, Sensitivity, Specificity e F1 a todos os modelos, onde iterando dez vezes com amostras diferentes da mesmo população, criando assim vetores de que nos permitiram no fiz fazer uma média de todos os resultados para as medidas de desempenho. Com estas medidas é nos possível no fim fazer uma escolha do melhor algoritmo mediamente o que seja mais crítico para nós.

\end{itemize}